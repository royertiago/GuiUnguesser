\documentclass{article}
\usepackage[T1]{fontenc}
\usepackage[utf8]{inputenc}
\usepackage[brazil]{babel}

\usepackage{listings}
\usepackage{tikz}
\usetikzlibrary{positioning}
\usetikzlibrary{shapes}

\newcommand{\ask}{\lstinline"ASK_QUESTION"}
\newcommand{\guess}{\lstinline"GUESS_ANSWER"}
\newcommand{\restart}{\lstinline"RESTART"}
\newcommand{\giveup}{\lstinline"GIVE_UP"}
\newcommand{\newquestion}{\lstinline"ASK_NEW_QUESTION"}
\newcommand{\unguesser}{\lstinline"Unguesser"}

\lstset{
    basicstyle=\ttfamily
}

\begin{document}

\title{
    INE5430 --- Inteligência Artificial \\
    Unguesser: jogo de adivinhações
}
\author{
    \begin{tabular}{r l}
        Lucas Berri Cristofolini & 12100757 \\
        Tiago Royer & 12100776 \\
        Wagner Fernando Gascho & 12100779
    \end{tabular}
}

\maketitle

\section{Máquina de Estados}

A classe principal do programa, \unguesser,
controla a ordem das operações.
É esta classe que escolha qual a próxima pergunta a ser feita,
quando e quem será chutado,
e quando desistir ou pedir ajuda.

Esta classe é melhor entendida como o autômato finito
descrito da figura \ref{unguesser-automata}.

\begin{figure}[h]
    \centering
    \begin{tikzpicture}[-stealth]
        \begin{scope}[every node/.style={shape=ellipse,draw}]
            \node (ask) {\ask};
            \node (guess) [below = of ask] {\guess};
            \node (giveup) [right = of guess] {\giveup};
            \node (newquestion) [below = of giveup] {\newquestion};
            \node (restart) [below = of guess] {\restart};
        \end{scope}

        \draw (ask) to[loop above] (ask);
        \draw (ask) to[bend left] (guess);
        \draw (guess) to[bend left] (ask);
        \draw (guess) to (giveup);
        \draw (guess) to (restart);
        \draw (giveup) to (newquestion);
        \draw (giveup) to (restart);
        \draw (newquestion) to (restart);
        \draw (node cs:name=restart, angle=180) to [bend left = 45]
              (node cs:name=ask, angle=185);
    \end{tikzpicture}

    \caption{Máquina de estados que corresponde à classe \unguesser.}
    \label{unguesser-automata}
\end{figure}

\end{document}
